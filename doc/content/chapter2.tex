
\chapter{Data structures}

This chapter will go through the most common data structures and how they are implemented in KamilaLisp. Many data structures in KamilaLisp are defined in terms of each other - for instance, all the functions that generally operate on sets can also be used on lists, since sets constitute a special case of lists. This chapter will also describe elementary array programming in KamilaLisp, which is usually the preferred by many programmers way to solve complex problems quickly.

\section{Basic list operations}

Lists are one of the most important data structures used in functional programming and every major programming language provides means of finite sequence storage. KamilaLisp is no different and a special emphasis is put on list and array processing, the basic building block of dataflow programming.

As mentioned before, every list besides the empty list contains a \textit{head} (the first element, \verb|car|) and a \textit{tail} (the last element, \verb|cdr|). The tail of a list is always another list, even if it is empty. The empty list literal is introduced in the code as \verb|'nil| or \verb|'()|. 

Using \verb|car| and \verb|cdr| it is possible to define a basic, non-tail recursive function that yields the length of a list:

\begin{Verbatim}
    --> defun length l (if (same l 'nil) 0 (+ 1 (length (cdr l))))
    (λ l . (if (same l 'nil) 0 (+ 1 (length (cdr l)))))
\end{Verbatim}

A generalised version of this function that handles scalar values, variadic application and strings is available as \verb|tally|\footnote{\textit{tally} - to count or calculate something}:

\begin{Verbatim}
    --> tally '(1 2 3) '(4 5)
    (3 2)
    --> tally '(1 2 3 4 5)
    5
    --> tally "abcde"
    5
    --> tally 5
    1
    --> tally
    0
\end{Verbatim}

Individual elements may be prepended to a list using the \verb|cons| function:

\begin{Verbatim}
    --> cons 6 'nil
    (6)
    --> cons 5 (cons 6 'nil)
    (5 6)
    --> cons 1 '(2 3)
    (1 2 3)
\end{Verbatim}

Hence, one could define a countdown function as follows:

\begin{Verbatim}
    --> defun countdown x (if x (cons x (countdown (- x 1))) '(0))
    (λ x . (if x (cons x (countdown (- x 1))) '(0)))
    --> countdown 5
    (5 4 3 2 1 0)
\end{Verbatim}

Once again, a general result of this function is available in KamilaLisp as the \verb|range| function:

\begin{Verbatim}
    --> range 5
    (0 1 2 3 4)
    --> range 5 10
    (5 6 7 8 9)
    --> range 10 5
    (10 9 8 7 6)
    --> range 5 -5
    (5 4 3 2 1 0 -1 -2 -3 -4)
\end{Verbatim}

List concatenation in KamilaLisp is accomplished using the \verb|append| function. The \verb|append| function is of course variadic and accepts an empty parameter list:

\begin{Verbatim}
    --> append '(1 2 3) '(4 5)
    (1 2 3 4 5)
    --> append '(1 2 3) '(4 5) '(6 7)
    (1 2 3 4 5 6 7)
    --> append
    'nil
    --> append 'nil 'nil
    'nil
    --> append "Tomato" "sauce"
    Tomatosauce
\end{Verbatim}

Prefixes and suffixes of lists may be extracted using the \verb|take| and \verb|drop| functions as follows:

\begin{Verbatim}
    --> take 3 '(1 2 3 4 5)
    (1 2 3)
    --> drop 3 '(1 2 3 4 5)
    (4 5)
    --> take 3 '(1 2 3)
    (1 2 3)
    --> drop 3 '(1 2 3)
    --> take 3 'nil
    [[]
     []
     []]
    --> drop 3 'nil
    --> take 5 '(1 2 3)
    (1 2 3 0 0)
\end{Verbatim}

The \verb|take| and \verb|drop| functions also accept negative argument, which changes the direction of the operation:

\begin{Verbatim}
    --> take -3 "KamilaLisp is Fun"
    Fun
\end{Verbatim}

More generally, \textit{all} prefixes and suffixes of a list are extracted using the \verb|prefixes| and \verb|suffixes| functions:

\begin{Verbatim}
    --> prefixes '(1 2 3 4 5)
    ((1) (1 2) (1 2 3) (1 2 3 4) (1 2 3 4 5))
    --> suffixes '(1 2 3 4 5)
    ((1 2 3 4 5) (2 3 4 5) (3 4 5) (4 5) (5))
    --> suffixes "Lisp"
    ("Lisp" "isp" "sp" "p")
\end{Verbatim}

Going back to the \verb|take| function, it is easy to notice that when the list is shorter than expected, the resultant list is simply padded with zeroes. This may not be the desired behaviour, thus a variant of \verb|take| called \verb|cycle| is provided. The \verb|cycle| function takes a list and a number and returns a list of the same length as the number, where the elements are taken from the list in a cyclic manner:

\begin{Verbatim}
    --> cycle 5 '(1 2 3)
    (1 2 3 1 2)
    --> cycle 3 '(1 2 3)
    (1 2 3)
    --> cycle 2 '(1 2 3)
    (1 2)
    --> cycle 1 '(1 2 3)
    (1)
    --> cycle 0 '(1 2 3)
    --> cycle -1 '(1 2 3)
    RuntimeException thrown in thread 1dbd16a6:
            cycle: negative length
        at entity cycle  1:1
        at cycle primitive function
    --> cycle 'nil 5
    --> cycle "abc" 5
    abcab
\end{Verbatim}

The \verb|replicate| function ubiquitously used in APL and Haskell is also available in KamilaLisp, except its domain is extended to scalar values:

\begin{Verbatim}
    --> replicate 3 5
    (5 5 5)
    --> replicate 5 '(1 2 3)
    (1 2 3 1 2 3 1 2 3 1 2 3 1 2 3)
    --> replicate 5 "Kamila"
    KamilaKamilaKamilaKamilaKamila
    --> replicate 0 5
    --> replicate 5 'nil
    --> replicate '(1 2 3) '(4 5 6)
    (4 5 5 6 6 6)
\end{Verbatim}

KamilaLisp also provides a few functions for altering the \textit{order} of elements in a list. The \verb|reverse| function reverses the order of elements in a list:

\begin{Verbatim}
    --> reverse '(1 2 3 4 5)
    (5 4 3 2 1)
    --> reverse "KamilaLisp"
    psiLalimaK
\end{Verbatim}

The \verb|rotate| function, as the name suggests, takes a list and a number and returns a list where the elements are rotated by the number:

\begin{Verbatim}
    --> rotate 2 '(1 2 3 4 5)
    (3 4 5 1 2)
    --> rotate -2 '(1 2 3 4 5)
    (4 5 1 2 3)
    --> rotate -4 "KamilaLisp"
    LispKamila
    --> rotate 1 'nil
    --> rotate 0 'nil
\end{Verbatim}

Finally, the \verb|list:shuffle| function will take a list and return a list with the same elements, but in a random order:

\begin{Verbatim}
    --> list:shuffle '(1 2 3 4 5)
    (2 5 3 1 4)
    --> list:shuffle "KamilaLisp"
    LpLsikmaia
\end{Verbatim}

\section{Sorting, searching and indexing}

KamilaLisp uses a special syntax for indexing into lists. The syntax is as follows:

\begin{Verbatim}
    --> def x '(1 5 2 3 4)
    (1 5 2 3 4)
    --> ?x$[0]
    1
\end{Verbatim}

This syntax is very confusing when demonstrated in isolation. First, indexing returns a value without a function call involved, so it is mandatory to tell the interpreter that the intent is to obtain the value of an object, hence the \verb|?| prefix in the REPL. It is possible to index a list using a list, as follows:

\begin{Verbatim}
    --> ?x$[1 3 4]
    (5 3 4)
\end{Verbatim}

The indexing function loops over the list it has received and returns respectively the first, third and fourth items of a list, all packed together into a single array. This is a very powerful feature, as it allows for a very concise syntax for extracting elements from a list, applying permutations and even sorting, as demonstrated later in the book. Of course, indexing can also be done using an expression:

\begin{Verbatim}
    --> ?x$[random 5]
    3
\end{Verbatim}

Another closely related functionality related to indexing is searching. The \verb|index-of| takes a value and a list and returns the index of the first occurrence of the value in the list:

\begin{Verbatim}
    --> index-of 5 '(9 8 6 5 4 7 2 3)
    3
    --> index-of 5 '(9 8 6 4 7 2 3)
    -1
\end{Verbatim}

\section{Rank}

KamilaLisp lists have \textit{rank}, which is a measure of their nesting, usually interpreted in the context of how many dimensions they could have. For example, a doubly nested list can be interpreted as a matrix:

\begin{Verbatim}
    --> ?'((1 2) (3 4))
    [[1 2]
     [3 4]]
\end{Verbatim}

Since a matrix usually has two axes, matrices (lists of lists of scalars) have rank 2. A vector is a list of scalar values, so it has rank 1. A scalar value has rank 0. The rank of an object can be computed using the \verb|rank| function:

\begin{Verbatim}
    --> rank '((1 2) (3 4))
    2
    --> rank '(1 2 3 4)
    1
    --> rank 0
    0
\end{Verbatim}

One interesting thing to consider are \textit{ragged list}. A \textit{ragged list} is defined as a list, whose elements have different ranks. For example, the following list is a ragged list:

\begin{Verbatim}
    --> ?'((1 2) (3 4) 5)
\end{Verbatim}

Since the first and second elements of it have ranks 1 (vectors; lists of scalars), the last element has rank 0 (a scalar). The rank of a ragged list is computed as if the maximum of ranks of a list was considered and the result is negated:

\begin{Verbatim}
    --> rank '((1 2) (3 4) 5)
    -2
\end{Verbatim}

\section{Elementary higher order functions}

KamilaLisp provides a wide range of higher order functions for manipulating lists. Many them can be defined using recursion, however almost all of them are guaranteed to terminate, while recursion in general case does not. The use of list processing functions that constitute the core of array programming is highly encouraged over recursion, because they tend to be more concise, easier to understand and less error prone.

The most used function is a built-in function takes a function and a list and applies the function to each element of the list, returning a list of the results. Define a successor funciton and map it over a list by prepending a single colon before the function name:

\begin{Verbatim}
    --> def s $(+ 1)
    $['+, '1]
    --> :s '(1 2 3 4 5)
    (2 3 4 5 6)
\end{Verbatim}

The \verb|map| function (which is the more familiar name of this construct, predominantly called that in Haskell and OCaml) has a few nuissances that are worth mentioning. First, it is possible to apply it to a non-list argument and an empty list:

\begin{Verbatim}
    --> :s 5
    (6)
    --> :s 'nil
    -->
\end{Verbatim}

The functor returned by \verb|:| has the same arity as the function it is applied to, hence it is possible for it to act as a \verb|zipWith| operation known from e.g. Haskell:

\begin{Verbatim}
    --> :+ '(1 2 3) '(4 5 6)
    (5 7 9)
    --> :+ '(1 2 3) '(4 5 6 7)
    (5 7 9)
    --> :+ '(1 2 3 4) '(4 5 6)
    (5 7 9)
    --> :+ '(1 1 1) '(2 2 2) '(3 3 3)
    (6 6 6)
\end{Verbatim}

Of course, it is possible to specify \textit{invariant arguments} to the function - the invariant arguments are constant arguments that are always supplied to the function being mapped, while other arguments over which \verb|map| can iterate are changing:

\begin{Verbatim}
    --> :+ 5 '(5) '(1 2 3 4 5)
    (11 12 13 14 15)
\end{Verbatim}

An important observation to be made is that it is possible to apply a function to a list of lists by stacking the comma operator multiple times also abusing the invariant arguments to write a function that forms tuples from the elements in a two-dimensional matrix:

\begin{Verbatim}
    --> def mat '((4 3) (3 4))
    [[4 3]
     [3 4]]
    --> ::cons 5 mat
    (((5 4) (5 3)) ((5 3) (5 4)))
\end{Verbatim}

The colon operator may not be general enough to be suitable for all uses. For example, it may be desirable to create a \textit{pervasive function} - a function which automatically applies itself to all the scalar values in a list. The built-in functions such as \verb|+|, \verb|-| or \verb|ln| are pervasive by default, but for example the \verb|reverse| function is not:

\begin{Verbatim}
    --> reverse '(("hi" "hello") ("kamila" "lisp"))
    (("kamila" "lisp") ("hi" "hello"))
\end{Verbatim}

Since strings are generally considered scalar values by KamilaLisp (however, this is not the case in other array programming languages such as APL), applying it \textit{on depth zero} yields the following results:

\begin{Verbatim}
    --> reverse%[0] '(("KamilaLisp" "is") "fun!")
    (("psiLalimaK" "si") "!nuf")
\end{Verbatim}

The function \verb|reverse| was ran on every object of rank zero of the list. If it is desirable reverse vectors (lists of scalars), the \verb|reverse| function should be applied \textit{on depth one}:

\begin{Verbatim}
    --> reverse%[1] '((1 2) 3 4)
    ((2 1) 3 4)
\end{Verbatim}

To give another example, to reverse the rows of a list of matrices, the function should be applied \textit{on depth two}:

\begin{Verbatim}
    --> reverse%[2] '(((1 2) (3 4)) ((5 6) (7 8)))
    ((2 1) (4 3))
\end{Verbatim}

Of course, since the depth operator is a \textit{generalisation} of the colon operator (mapping), it is possible to use it to map a function over a list of lists. In this case, the depth specifier must be negative:

\begin{Verbatim}
    --> writeln%[-1] '("Hello" "world!")
\end{Verbatim}

The smaller negative number, the more times the map function is applied:

\begin{Verbatim}
    --> cons%[-2] mat 5
    (((4 5) (3 5)) ((3 5) (4 5)))
    --> ::cons mat 5
    (((4 5) (3 5)) ((3 5) (4 5)))
\end{Verbatim}

An important thing to note is that the depth operator subtly differs from the colon operator in the variadic case. The colon operator will determine the shape of the result ad-hoc, regardless of argument order:

\begin{Verbatim}
    --> ::cons mat 5
    (((4 5) (3 5)) ((3 5) (4 5)))
    --> ::cons 5 mat
    (((5 4) (5 3)) ((5 3) (5 4)))
\end{Verbatim}

The depth operator, however, will always infer the shape from its first argument, potentially leading to unexpected results:

\begin{Verbatim}
    --> cons%[-2] mat 5
    (((4 5) (3 5)) ((3 5) (4 5)))
    --> cons%[-2] 5 mat
    [[5 4]]
\end{Verbatim}

This behaviour significantly differs from the behaviour of the depth operator in other languages, such as APL\footnote{https://aplwiki.com/wiki/Depth\_(operator)}, where the depth operator is restricted to only two arguments, making it feasible to try dynamically determining the shape of the result. In KamilaLisp the shape is inferred from the first argument, since the operator's complexity would grow by a large magnitude as a result of it being a generalisation to an arbitrary amount of arguments. Additionally, the complex inferring rule in APL-like languages is not very useful in practice and leads to some design shortcomings.

To explore this topic further, it is necessary to demonstrate that the KamilaLisp depth operator accepts multiple depth values. For example, the following program will apply the function to objects of rank one extracted from the first array, and the objects obtained by descending once into the second array:

\begin{Verbatim}
    --> defun f (x y) (str:format "{?x}, {?y}")
    (λ x y . (str:format "{?x}, {?y}"))
    --> f%[1 -1] '((1 2) (3 4)) '(6 5 (4 3) 2 1)
    ("(1 2), 6" "(3 4), 5")
\end{Verbatim}

Notice that the depth operator makes an attempt to salvage the situation arising due to the fact that the lists are of different sizes by trimming the longer list to the size of the shorter one (which is not done by APL). To make a more fair comparison with APL, consider the following program instead:

\begin{Verbatim}
    --> f%[-2 0] '((1 2) 3) '((1 2) 3)
    (("1, 1" "2, 2") "3, 3")
\end{Verbatim}

Since the depth operator simply extracts the objects of the specified rank from the arguments, it does not pay attention to the shape of other arguments, so scalars for the second argument to f can be extracted also in this case:

\begin{Verbatim}
    --> f%[-2 0] '((1 2) 3) '(1 2 3)
    (("1, 1" "2, 2") "3, 3")
\end{Verbatim}

Since APL determines the shape in a more "clever" (also way slower and more convoluted) way, this behaviour can not be achieved:

\begin{Verbatim}
        ((1 2) 3) ({⍺⍵} ⍥ ¯2 0) (1 2) 3
┌─────────┬───┐
│┌───┬───┐│3 3│
││1 1│2 2││   │
│└───┴───┘│   │
└─────────┴───┘
        ((1 2) 3) ({⍺⍵} ⍥ ¯2 0) 1 2 3
LENGTH ERROR
        ((1 2)3)({⍺ ⍵} ⍥ ¯2 0)1 2 3
                ∧
\end{Verbatim}

Finally, the depths list can be defined as a result of an expression:

\begin{Verbatim}
    --> f%[[tie + -] 1] '((1 2) (3 4)) '(6 5 (4 3) 2 1)
    ("(1 2), 6" "(3 4), 5")
\end{Verbatim}

A function similar to list mapping is \verb|filter|. It takes a predicate and a list and returns a list of elements for which the predicate returned a truthy value.

\begin{Verbatim}
    --> filter (lambda x (> x 3)) '(1 2 3 4 5)
    (4 5)
\end{Verbatim}

The relation between \verb|filter| and \verb|map| (the colon operator) can be observed by re-implementing one in terms of the other in the following way:

\begin{Verbatim}
    --> defun my-filter (pred lst) (replicate (= 1 (:pred lst)) lst)
    (λ pred lst . (replicate (= 1 (:pred lst)) lst))
    --> my-filter (lambda x (> x 3)) '(1 2 3 4 5)
    (4 5)
\end{Verbatim}

\verb|filter| allows for an elegant yet inefficient implementation\footnote{Compared to the prime number-related primitive functions already supplied by KamilaLisp} of a prime sieve. The sieve of Eratosthenes is an algorithm for finding all prime numbers up to a given limit. It works by iteratively marking the multiples of each prime number as composite. To implement this in KamilaLisp, it is needed to use recursion and two lists of numbers: one for the primes that have been already found, and one for the numbers that have not been classified yet.

Firstly, define the iteration step function that takes the prime and unclassified lists in a pair, adds the first element from the unclassified list to the prime list and removes all its multiples from the unclassified list:

\begin{Verbatim}
    --> defun step data (let-seq (
        (def primes (car data))
        (def cands (car@cdr data))
        (case (same cands 'nil) (tie primes 'nil))
        (def current (car cands))
        (tie (cons current primes) (filter $(mod _ current) (cdr cands)))))
\end{Verbatim}

This example of \verb|let-seq| used the construct \verb|case|, which is a special form of \verb|if| that may be used only in \verb|let-seq|. If the condition that directly follows \verb|case| is true, then the further execution of the \verb|let-seq|'s body is stopped and the value of the expression that follows the condition is returned. This is especially useful for terminating the computation when a special case is encountered. Using \verb|while|, the iteration step function can be applied a finite amount of times to the initial pair of lists:

\begin{Verbatim}
    --> while (tie 'nil (range 2 50)) 1 step
    ((2) (3 5 7 9 11 13 15 17 19 21 23 25 27 29 31 33 35 37 39 41 43 45 47 49))
    --> while (tie 'nil (range 2 50)) 5 step
    ((11 7 5 3 2) (13 17 19 23 29 31 37 41 43 47))
    --> while (tie 'nil (range 2 50)) 10 step
    ((29 23 19 17 13 11 7 5 3 2) (31 37 41 43 47))
    --> while (tie 'nil (range 2 50)) 15 step
    ((47 43 41 37 31 29 23 19 17 13 11 7 5 3 2) nil)
    --> while (tie 'nil (range 2 50)) 20 step
    ((47 43 41 37 31 29 23 19 17 13 11 7 5 3 2) nil)
\end{Verbatim}

Since the step function \textit{converges} to a value (eventually yields a value $t$ such that $f(t) = t$), the \verb|converge| function can be used to implement most of the sieve's logic now:

\begin{Verbatim}
    --> converge step (tie 'nil (range 2 100))
    ((97 89 83 79 73 71 67 61 59 53 47 43 41 37 31 29 23 19 17 13 11 7 5 3 2) nil)
\end{Verbatim}

To provide a final result, it is necessary to extract and reverse the prime list from the pair and wrap the invocation in a function:

\begin{Verbatim}
    --> defun sieve (n) (reverse@car@converge step (tie 'nil (range 2 n)))
    (λ n . (reverse@car@converge step (tie 'nil (range 2 n))))
    --> sieve 20
    (2 3 5 7 11 13 17 19)
\end{Verbatim}

Ultimately, the function can be rewritten as follows to make it more concise and self-contained:

\begin{Verbatim}
defun sieve (n) (reverse@car@converge (
        lambda x (let-seq (
            (def primes (car x))
            (def cands (car@cdr x))
            (case (same cands 'nil) (tie primes 'nil))
            (def current (car cands))
            (tie (cons current primes) (filter $(mod _ current) (cdr cands)))))
    ) (tie 'nil (range 2 n)))
\end{Verbatim}

\section{Folding and scanning}

\section{Searching and partitioning}

\section{Sorting and permutations}

\section{Pattern matching}

\section{Sets}

\section{Hashmaps}

\section{Using glyphs}
