
\chapter{List processing}

Lists are one of the most important data structures used in functional programming and every major programming language provides means of finite sequence storage. KamilaLisp is no different and a special emphasis is put on list and array processing, the basic building block of dataflow programming.

As mentioned before, every list besides the empty list contains a \textit{head} (the first element, \verb|car|) and a \textit{tail} (the last element, \verb|cdr|). The tail of a list is always another list, even if it is empty. The empty list literal is introduced in the code as \verb|'nil| or \verb|'()|. 

\section{Basic list operations}

Using \verb|car| and \verb|cdr| it is possible to define a basic, non-tail recursive function that yields the length of a list:

\begin{Verbatim}
    --> defun length l (if (same l 'nil) 0 (+ 1 (length (cdr l))))
\end{Verbatim}

A generalised version of this function that handles scalar values, variadic application and strings is available as \verb|tally|:

\begin{Verbatim}
    --> tally '(1 2 3) '(4 5)
    (3 2)
    --> tally '(1 2 3 4 5)
    5
    --> tally "abcde"
    5
    --> tally 5
    1
    --> tally
    0
\end{Verbatim}
