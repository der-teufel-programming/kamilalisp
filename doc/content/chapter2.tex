
\chapter{List processing}

Lists are one of the most important data structures used in functional programming and every major programming language provides means of finite sequence storage. KamilaLisp is no different and a special emphasis is put on list and array processing, the basic building block of dataflow programming.

As mentioned before, every list besides the empty list contains a \textit{head} (the first element, \verb|car|) and a \textit{tail} (the last element, \verb|cdr|). The tail of a list is always another list, even if it is empty. The empty list literal is introduced in the code as \verb|'nil| or \verb|'()|. 

\section{Basic list operations}

Using \verb|car| and \verb|cdr| it is possible to define a basic, non-tail recursive function that yields the length of a list:

\begin{Verbatim}
    --> defun length l (if (same l 'nil) 0 (+ 1 (length (cdr l))))
    (λ l . (if (same l 'nil) 0 (+ 1 (length (cdr l)))))
\end{Verbatim}

A generalised version of this function that handles scalar values, variadic application and strings is available as \verb|tally|\footnote{\textit{tally} - to count or calculate something}:

\begin{Verbatim}
    --> tally '(1 2 3) '(4 5)
    (3 2)
    --> tally '(1 2 3 4 5)
    5
    --> tally "abcde"
    5
    --> tally 5
    1
    --> tally
    0
\end{Verbatim}

Individual elements may be prepended to a list using the \verb|cons| function:

\begin{Verbatim}
    --> cons 6 'nil
    (6)
    --> cons 5 (cons 6 'nil)
    (5 6)
    --> cons 1 '(2 3)
    (1 2 3)
\end{Verbatim}

Hence, one could define a countdown function as follows:

\begin{Verbatim}
    --> defun countdown x (if x (cons x (countdown (- x 1))) '(0))
    (λ x . (if x (cons x (countdown (- x 1))) '(0)))
    --> countdown 5
    (5 4 3 2 1 0)
\end{Verbatim}

Once again, a general result of this function is available in KamilaLisp as the \verb|range| function:

\begin{Verbatim}
    --> range 5
    (0 1 2 3 4)
    --> range 5 10
    (5 6 7 8 9)
    --> range 10 5
    (10 9 8 7 6)
    --> range 5 -5
    (5 4 3 2 1 0 -1 -2 -3 -4)
\end{Verbatim}

List concatenation in KamilaLisp is accomplished using the \verb|append| function. The \verb|append| function is of course variadic and accepts an empty parameter list:

\begin{Verbatim}
    --> append '(1 2 3) '(4 5)
    (1 2 3 4 5)
    --> append '(1 2 3) '(4 5) '(6 7)
    (1 2 3 4 5 6 7)
    --> append
    'nil
    --> append 'nil 'nil
    'nil
    --> append "Tomato" "sauce"
    Tomatosauce
\end{Verbatim}

Prefixes and suffixes of lists may be extracted using the \verb|take| and \verb|drop| functions as follows:

\begin{Verbatim}
    --> take 3 '(1 2 3 4 5)
    (1 2 3)
    --> drop 3 '(1 2 3 4 5)
    (4 5)
    --> take 3 '(1 2 3)
    (1 2 3)
    --> drop 3 '(1 2 3)
    --> take 3 'nil
    [[]
     []
     []]
    --> drop 3 'nil
    --> take 5 '(1 2 3)
    (1 2 3 0 0)
\end{Verbatim}

The \verb|take| and \verb|drop| functions also accept negative argument, which changes the direction of the operation:

\begin{Verbatim}
    --> take -3 "KamilaLisp is Fun"
    Fun
\end{Verbatim}

More generally, \textit{all} prefixes and suffixes of a list are extracted using the \verb|prefixes| and \verb|suffixes| functions:

\begin{Verbatim}
    --> prefixes '(1 2 3 4 5)
    ((1) (1 2) (1 2 3) (1 2 3 4) (1 2 3 4 5))
    --> suffixes '(1 2 3 4 5)
    ((1 2 3 4 5) (2 3 4 5) (3 4 5) (4 5) (5))
    --> suffixes "Lisp"
    ("Lisp" "isp" "sp" "p")
\end{Verbatim}
