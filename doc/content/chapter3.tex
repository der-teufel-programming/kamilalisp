
\chapter{Functional data structures}

This chapter describes more intricate details regarding more complex data structures, their representations in functional programs and their applications in the context of real-world problems.

\section{Church and Scott-Morgensen encodings}

In mathematics, Church encoding is a means of representing data and operators in the lambda calculus.

The Church numerals are the representations of natural numbers under Church encoding. They can be easily defined in terms of iterated function composition:

$$
\begin{array}{r|l}
    {\text{Number}}&{\text{Function definition}}\\
    \hline
    0&0\ f\ x=x\\
    1&1\ f\ x=f\ x\\
    2&2\ f\ x=f\ (f\ x)\\
    3&3\ f\ x=f\ (f\ (f\ x))\\
    \vdots&\vdots \\
    n&n\ f\ x=f^{n}\ x
\end{array}
$$

This concept can be trivially translated into KamilaLisp. For example, the Church numerals for 0 and 3 are given by:

\begin{Verbatim}
    --> defun c0 f #0
    (λ f . #0)
    --> defun c3 f (λ x (f (f (f x))))
    (λ f . (λ x (f (f (f x)))))
\end{Verbatim}

To verify the correctness of this definition and further experiments with Church numerals, the following function is defined to convert an arbitrary Church-encoded numeral into a natural number:

\begin{Verbatim}
    --> defun nat f ((f $(+ 1)) 0)
    (λ f . ((f $(+ 1)) 0))
    --> nat c3
    3
\end{Verbatim}

This function utilises the fact that Church encoding is in fact identical to iterated function composition, hence the natural successor function \verb|$(+ 1)| can be applied to it with a starting value of \verb|0|.

To obtain the Church numeral for any natural number, define the following \textit{successor} function:

\begin{Verbatim}
    --> defun succ x (λ f (λ a (f ((x f) a))))
    (λ x . (λ f (λ a (f ((x f) a)))))
\end{Verbatim}

Using these, it is possible to verify that \verb|succ(succ(0)) = 2|:

\begin{Verbatim}
    --> nat (succ (succ c0))
    2
\end{Verbatim}

Using these properties, it is possible to define a function that yields the Church-encoded numeral for any given natural number:

\begin{Verbatim}
    --> defun church n (if n (succ (&0 (- n 1))) c0)
    (λ n . (if n (succ (&0/syn (- n 1))) c0))
    --> nat (church 5)
    5
\end{Verbatim}

Before giving more examples of operations on Church numerals, it is important to point out that using the helper functions \verb|church| and \verb|nat| would completely defeat the purpose of Church encoding in the first place, hence they will be used only to verify concrete results, and not to define new functions.

\section{Elias \texorpdfstring{$\gamma$}{gamma} coding}

\section{Sets}

\section{Queues}

\section{Hashmaps}

\section{Trees}

\section{Graphs}
